
\documentclass[10pt]{beamer}
%\documentclass[10pt,handout]{beamer}
%\documentclass{beamer}

\usepackage{amsmath}
\usepackage{amssymb}
\usepackage{latexsym}
\usepackage{epsfig}
\usepackage{graphics}


\mode<presentation>
{
\usetheme{Warsaw}
%\setbeamercovered{dynamic}
\setbeamercolor{palette primary}{bg=blue}
\setbeamercolor{block title}{bg=blue}
\setbeamercolor{lowercol}{bg=green}
}
\setbeamertemplate{navigation symbols}{} 

\pgfdeclareimage[height=.9cm]{AULogo2}{SampTA2011Figs/AULogo2} \logo{\pgfuseimage{AULogo2}}

%==========================(Casey's LaTeX Shortcuts)================================
%===================================================================================

% SYMBOLS %
% ---------------------- Blackboard Bold --------------------------------
\def\R{\mathbb{R}}                     % real numbers R
\def\C{\mathbb{C}}                     % complex numbers C
\def\N{\mathbb{N}}                     % natural numbers
\def\Z{\mathbb{Z}}                     % integers
\def\Q{\mathbb{Q}}                     % rational numbers
\def\P{\mathbb{P}}                     % primes
\def\W{\mathbb{W}}                     % windowing functions
\def\B{\mathbb{B}}                     % partition of unity functions
\def\A{\mathbb{A}}                     % almost orthogonal windowing functions
\def\T{\mathbb{T}}                     % unit circle (\R mod 2 \pi)
\def\F{\mathbb{F}}                     % Functions
\def\I{\mathbb{I}}                     % Intervals
\def\D{\mathbb{D}}                     % The unit disk
\def\Rn{{\R}^{n}}                      % n-dimensional R space
\def\Cn{{\C}^{n}}                      % n-dimensional C space
\def\Rhat{{\widehat{\R}}}              % reals (dual)
\def\Rnhat{{\widehat{\Rn}}}            % n-dim reals (dual)
\def\PW{\mathbb{PW}}                   % Paley-Wiener Space
 
% WORD FUNCTIONS %
\def\sinc{\mathop{\textstyle{\rm sinc}}}
\def\domain{\mathop{\textstyle{\rm Domain}}\nolimits}
\def\essinf{\mathop{\textstyle{\rm ess \, inf}}}
\def\esssup{\mathop{\textstyle{\rm ess \, sup}}}
\def\range{\mathop{\textstyle{\rm Range}}\nolimits}
\def\Span{\mathop{\textstyle{\rm span}}\nolimits}
\def\supp{\mathop{\textstyle{\rm supp}}\nolimits}
\def\CHI{\hbox{\raise .5ex \hbox{$\chi$}}}
\def\Tri{\mbox{Tri}}
\def\Trap{\mbox{Trap}}
\def\Cap{\mbox{Cap}}
\def\loc{{\textstyle{\rm loc}}}
\def\qed{ \hspace*{\fill} \Box}

% UNARY, BINARY OPERATORS %
\def\norm#1{\|  #1 \|}
\def\abs#1{| #1 |}
\def\bigabs#1{\biggl| #1 \biggr|}
\def\set#1{\{ #1 \}}
\def\bigset#1{\biggl\{ #1 \biggr\}}
\def\ceiling#1{\lceil #1 \rceil}
\def\floor#1{\lfloor #1 \rfloor}
\def\ip#1#2{\langle #1 , #2 \rangle}
\def\bigip#1#2{\bigl\langle #1, \, #2 \bigr\rangle}
\def\Bigip#1#2{\Bigl\langle #1, \; #2 \Bigr\rangle}
\def\biggip#1#2{\biggl\langle #1, \; #2 \biggr\rangle}
\def\ipsize#1#2{\left\langle #1 , #2 \right\rangle}
\def\Int#1{\lfloor #1 \rfloor}
\def\biggInt#1{\biggl\lfloor #1 \biggr\rfloor}
\def\Intsize#1{\left\lfloor #1 \right\rfloor}
\def\paren#1{( #1 )}
\def\bigparen#1{\left( #1 \right)}
\def\Bigparen#1{\Bigl( #1 \Bigr)}
\def\biggparen#1{\biggl( #1 \biggr)}
\def\Biggparen#1{\Biggl( #1 \Biggr)}
\def\sqparen#1{[ #1 ]}
\def\bigsqparen#1{\biggl[ #1 \biggr]}

\newcommand{\hl}[1]{{\bf #1}}
\newcommand{\refp}[1]{(\ref{#1})}

\def\comment#1{}

%==========================(End of my LaTeX Commands)===============================
%

%\begin{document}

\title{
{\bf{\textsf{Math Hacks 2}}}}
\author[Erik Taubeneck]{Erik Taubeneck}

\institute[Hacker School - Summer 2013] 
{
{\tt erik.taubeneck@gmail.com}
}

\date{August 8th, 2013}
 
\begin{document}

% Creates title page of slide show using above information
\begin{frame}
  \titlepage
\end{frame}
\note{Talk for 5 minutes} % Add notes to yourself that will be displayed when
                           % typeset with the notes or notesonly class options

%\section[Outline]{}

% Creates table of contents slide incorporating
% all \section and \subsection commands

\begin{frame}
\frametitle{A Crash Course in Complex Analysis}

\begin{itemize}[<+->]
  \item Def: $i$ = $\sqrt{-1}$.
  \item Why do we need $i$???
  \item Solve $x^2 + 1 = 0$
  \item Def: $\C = \{x + iy \ \forall \ x,y \ \in \R^2 \}$
  \item Fun fact! All polynomials of degree $n$ have $n$ zeros in $\C$.
\end{itemize}


\end{frame}

\begin{frame}
\frametitle{A Crash Course in Complex Analysis (cont.)}

$\mathbb{E}^x$:
\[ f(z) = z^2 \]
\pause \uncover{
  Let $z=x+iy$. Then

  \[ f(z) = f(x+iy) = (x+iy)^2 = x^2 +2ixy + i^2y^2 = \]
  \[ x^2+ 2ixy - y^2 \]
}

\pause \uncover{
  We can always write $f(x+iy) = u(x,y) + iv(x,y)$. Note that

  \[ u(x,y) = \mathfrak{R}(f(x+iy)) \ \mathrm{and} \ v(x,y) = \mathfrak{I}(f(x+iy)) \]
}
\pause \uncover{
  In our example

  \[ u(x,y) = x^2 - y^2 \ \mathrm{and} \ v(x,y) = 2xy \]

}
\end{frame}

\begin{frame}
\frametitle{A Crash Course in Complex Analysis (cont.)}

\[ u(x,y) = x^2 - y^2 \ \mathrm{and} \ v(x,y) = 2xy \]

We can now take 4 different derivatives.

\begin{eqnarray*}
\frac{\partial u}{\partial x} = 2x && \frac{\partial v}{\partial x} = 2y \\
\frac{\partial u}{\partial y} = -2y && \frac{\partial v}{\partial y} = 2x
\end{eqnarray*}

\pause \uncover{
  Note that:
  \begin{eqnarray*}
  \frac{\partial u}{\partial x} = \frac{\partial v}{\partial y} &&
  \frac{\partial u}{\partial y} = -\frac{\partial v}{\partial x}
  \end{eqnarray*}
  These are the Cauchy-Riemann equations and hold for all analytic functions on $\C$!
}
\end{frame}

\begin{frame}
\frametitle{Time for the Hack!}

Suppose we have a function $f:\R \to \R$ that we wish to find the derivative for a $x_0$.

\pause \uncover{
  One approximation is to use the finite difference method. Recall that $\frac{df}{dx}$ is defined as

  \[ \lim_{h \to 0} \frac{f(x+h)-f(x)}{h} \]

  So, we estimate

  \[ f'(x_o) \approx \frac{f(x_0+h)-f(x_0)}{h} \]

  for some small $h$.

}
\pause \uncover{
  So, what's wrong with that? Well for small differences, $\mathbf{Float64} - \mathbf{Float64}$ can have issues.
}

\end{frame}

\begin{frame}
\frametitle{Time for the Hack! (cont.)}

  We have $f:\R \to \R$. Rewrite

  \[ f(z) = f(x+iy) = u(x,y) + iv(x,y) \]
\pause \uncover{
  Now, for all $\bar{z} \in \R$, we have $y = 0$, and so

  \[ f(\bar{z}) = f(x,0) = u(x,0) + iv(x,0) \]
}
\pause \uncover{
  Since $f: \R \to \R$, $f(\bar{z}) \in \R$ for all $\bar{z} \in \R$. Therefore

  \[ v(x,0) = 0 \]

  and

  \[ f(x,0) = u(x,0) \]
}
\pause \uncover{
  We want to estimate $\frac{df}{\partial x}$. Since for all $z \in \R$, $f = u$,

  \[ \frac{df}{\partial x} = \frac{\partial u}{\partial x} = \frac{\partial v}{\partial y} \]

}

\end{frame}

\begin{frame}
\frametitle{Time for the Hack! (cont.)}

  Again, using the definition of the derivative

  \[ \frac{\partial v}{\partial y} = \lim_{h \to 0} \frac{v(x,y+h) - v(x,y)}{h} \]

\pause \uncover{

  Since $y = 0$ for all $\bar{z} \in \R$,

  \[ \frac{\partial v}{\partial y} = \lim_{h \to 0} \frac{v(x,h) - v(x,0)}{h} \]
}
\pause \uncover{
  
  Recall that $v(x,0) = 0$, so

  \[ \frac{\partial v}{\partial y} = \lim_{h \to 0} \frac{v(x,h)}{h} \]
}
\pause \uncover{
  Now, to estimate $\frac{df}{\partial x} = \frac{\partial v}{\partial y}$, for a very small $h$

  \[ \frac{df}{\partial x} \approx \frac{v(x,h)}{h} = \frac{\mathfrak{Im}(f(x+ih))}{h} \]
}

\end{frame} 
\end{document}
